\documentclass{Head}
\begin{document}
\tableofcontents
\linenumbers
\section{Introduction}
Macromolecules are characterized by their long-chain structure, including molecular chain unit and conformation on angstrom scale, lamella on nanometer scale and spherulites on micron scale.
Nowadays, synchrotron radiation small angle X-ray scattering (SAXS) and wide angle X-ray diffraction (WAXD), as a non-destructive, highly statistically averaged structure analysis method, have been widely used in crystalline polymer research area.
For instance, study information on grains in crystalline polymer, micro-domains in blended polymers and the shape, size and distribution of cavities and cracks can be obtained by guinier scattering.
Study information on orientation, thickness and crystalline fraction of crystalline layer and the thickness of amorphous layer of the lamella can be obtained by long-period measurement.


In order to further study the internal structure of polymers, two new test condition are required.
Firstly, considering the size of polymer spherulites, an X-ray spot with a size of $5\mu m \cdot 5\mu m$ is needed.
A small spot provides sufficient spatial resolution when the structure of macromolecules are characterized by the SAXS method.
Secondly, In order to match the detection result with the real structure, confirming the real-time exact position of X-ray incident beam on polymer crystall is a critical measure.


To improve the spatial resolution of synchrotron radiation experiment, the world’s advanced synchrotron radiation light sources all use micro-focusing to focus the spot to the micron or even sub-micron level.
Representative beamline stations include DSEY-PETRA \uppercase\expandafter{\romannumeral3} P03, SSRF BL15U1. However, limited by the distance between the sample and the detector, SAXS experiment can not be implemented on these stations.
\section{Experiment}
\section{Application}
\end{document}